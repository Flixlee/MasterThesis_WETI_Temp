% !TEX spellcheck = en_US
\chapter{Background} \label{cha:Back}
This chapter aims to give the necessary background information to follow the content of the thesis. In addition the Field of research is described the outcome of the thesis aims to enhance.
\section{Wind Turbine Control Systems}\label{sec:ControlSys}
Following \cite{Burton2011} the controller of a \gls{WT} uses inputs like generator speed, and wind measurement to generate outputs for actuators as the generator torque or the pitch angle. The whole control system contains the supervisory control, the safety control and the real-time closed-loop control.



\subsection{Feedback Control}\label{subsec:FB-Contr}









\subsection{Feedforward Control}\label{subsec:FF-Contr}





\section{Lidar Systems}\label{sec:LidarSys}
Lidar systems (\textit{light detection and ranging}) are part of the family of remote sensing technologies like radar and sonar. The first use of a lidar system was for measuring clouds and pollution as described in \cite{Goyer1963}. Since then the \gls{lidar} technology is used in several different fields from meteorology \cite{Goyer1963} to autonomous automotive sensors \cite{EDN2016}. This sections focuses on the \gls{lidar} systems in wind energy and there underlying measurement principles. The following sections are mainly based on \cite{DTUlidar2013} and \cite{Schlipf2015}.

\subsection{Lidar Systems in Wind Energy}\label{subsec:LidarInWind}
As described in Chapter \ref{cha:INT} mentioned, the driving force of a \gls{WT} is the wind speed. The \gls{lidar} system is used in wind energy to measure such. The commonly used systems can be characterized by: detection, waveband, waveform and position relative to the flow:

\begin{itemize}
	\item Direct or coherent detection: Systems that use coherent detection determine a Doppler shift by comparing the frequency of the transmitted light with the backscattered light frequency. This frequency difference is then converted into a wind speed. In contrast, systems with direct detection apply cross-correlation of signals backscattered from multiple measurement points to directly calculate wind speed and direction.
	
	\item Continuous or pulsed waveform: Pulsed lidar systems send out short laser pulses and can measure multiple distances at once by linking changes in the backscattered signal over time to spatial changes, taking into account the constant speed of light. These measured distances are often referred to as range gates. Continuous wave systems, however, must adjust the optical focus to measure at different distances.
	
	\item Ultra-violet or infra-red waveband: Lidar systems use different laser sources that produce either infrared light, which bounces off particles like dust, salt, or water droplets suspended in the air, or ultraviolet light, which is scattered by the air molecules themselves.
	
	\item Nacelle-based, ground-based or spinner-based systems: There are three main locations to install a \gls{lidar} system. Depending on that the systems have different data processing and measurement positions.
\end{itemize}
%
The \gls{lidar} system used in this thesis is the Molas NL 200 developed by Nanjing Movelaser \cite{MolasNL200}. The system uses coherent detection in the infra-red waveband by emitting laser pulses and is based on top of the nacelle.

\subsection{Measurement Concept of Nacelle mounted Doppler shift Lidar Systems}\label{subsec:LidarInWindConcept}
Due to the Doppler effect the backscattered light signal has a different frequency then the emitted light signal. The fundamental measurement principle of coherent detection is now to detect this frequency shift by comparing the backscattered signal to the emitted signal. To achieve this, part of the emitted light laser beam with the frequency \gls{symb:fL} is separated by a beam-splitter. This is than superimposed at a signal from another beam-splitter which contains the backscattered light respectively. The backscattered signal has the frequency $\gls{symb:fL} + \gls{symb:fD}$. Hence the source signal with frequency \gls{symb:fL} is superimposed the detector is able to detect the frequency shift \gls{symb:fD}.     

\noindent To process this signal into line of sight wind speed the detector signal is splitted into blocks. An \gls{FFT} is applied on each block. The resulting spectrum is then averaged over several blocks, this depends on the waveform. Continuous wave systems have a fixed range therefore the blocks can be averaged continuously. Pulsed systems measure at different range gates. The averaging need to be applied on different pulses at the same range gate. The process of averaging is necessary to ensure a clear distance between the measurement peak and noise. The metric to estimate the quality of the measurement is usually the \gls{cnr}. To finally get the detector output, the frequency shift \gls{symb:fD}, an peak detection algorithm is used.  

\noindent Hence we want to measure the wind speed and not a frequency the frequency shift needs to be translated. The result is then the wind speed in the trajectory of the laser beam the so called line-of-sight wind speed \gls{symb:v_LOS}. This is done with the wavelength \gls{symb:lambda_L} of the emitted laser light, the speed of light \gls{symb:c} and the frequency shift \gls{symb:fD} shown in equation \ref{eq:vLOSlidar}.
\begin{equation}
	\gls{symb:v_LOS} =  \frac{c f_\textnormal{D}}{2f_\textnormal{L}} = \frac{\lambda_\textnormal{L} f_\textnormal{D}}{2}
	\label{eq:vLOSlidar}	
\end{equation} 

\section{Wind Field Reconstruction}\label{sec:WindFieldReconstruction}
As described in the previous section \ref{subsec:LidarInWindConcept} the output of a \gls{lidar} system is a wind speed in line-of-sight. This has by its nature one main benefit and one main drawback that makes the \gls{WFR} necessary:
\begin{enumerate}
	\item The line-of-sight measurement is unable to measure a three-dimensional wind vector \cite{Schlipf2015}.
	\item The line-of-sight measurement is already measuring a probe volume instead of a point \cite{Schlipf2015}. Therefore it contains information averaged over space and time. 
\end{enumerate}
In the following Chapter !!!REF!!! in section !!!REF!!! it is explained in detail how a three-dimensional wind field is modeled and the effect it has on a \gls{WT}. To understand the importance of \gls{WFR} the most important step to understand is, that $\gls{symb:v_LOS} \neq \gls{symb:v_0}$. Where \gls{symb:v_0} is the \gls{REWS}, represented by one scalar that accounts for all three-dimensional wind vectors effecting the \gls{WT}. The \gls{WFR} is used to align the line-of-sight measurement in such a way that it is comparable to the \gls{REWS}. 
This chapter contains methods to reconstruct the measurement of a nacelle mounted \gls{lidar} into a \gls{REWS}. First the Static approach in \ref{subsec:StaticWFR} and a brief outlook on dynamic \gls{WFR} in section \ref{subsec:DynamicWFR}. This chapter is mainly based on \cite{Schlipf2011} and \cite{Schlipf2012}.

\subsection{Static Wind Field Reconstruction for nacelle lidar systems}\label{subsec:StaticWFR}
Es mentioned above the drawback of a \gls{lidar} measurement is, that it is not a three-dimensional vector. To get achieve this three \gls{lidar} systems are needed as in the Musketeer experiment \cite{Mikkelsen2008}. \cite{Schlipf2011} proposes that it can still be done by the following assumptions of \gls{WFR}:
\begin{enumerate}
	\item The wind only contains the longitudinal component \gls{symb:u}.
	\item The flow is homogeneous and the vertical wind component \gls{symb:z} is not contained in the wind field. 
\end{enumerate}
The first assumption is based on the idea that the \gls{WT} is perfectly align with the inflow wind in respect to the wind direction. Leading to the assumption of vertical shear if a two beam \gls{lidar} measures different \gls{symb:v_LOS} at the beams. The second assumption is used to detect a change in wind direction. Normally used if the \gls{lidar} is used for yaw-control. Only one or the other assumption of \gls{WFR} can be applied.

\noindent To reconstruct a \gls{REWS} from a line-of-sight measurement of a nacelle \gls{lidar} system one of the two above mentioned methods is selected. As described in !!!REF!!! this thesis aims to get e better estimation of the wind speed. The \gls{WFR} used for speed control according to \cite{Schlipf2011} is the first one. This leads to an estimated line-of-sight wind speed for one example laser beam as described in equation \ref{eq:simpleWFR}. Equation \ref{eq:simpleWFR} represents the simplified result presented in \cite{Schlipf2015}. Where $i$ refers to a beam corresponding to a specific measurement point and \gls{symb:thetaCL} the angle between the trajectory of the nacelle and the laser beam. 
\begin{equation}
	\gls{symb:v_LOS}_{,i} =  \hat{u}_i \cdot \cos(\gls{symb:thetaCL})
	\label{eq:simpleWFR}	
\end{equation}
Now the longitudinal wind speed at every beam focus point can be estimated by inserting the corresponding \gls{lidar} measurement \gls{symb:v_LOS} in equation \ref{eq:simpleWFR}. To get a value comparable to the \gls{REWS}, \gls{symb:v_0L} needs to be computed. This is done by averaging over all beams see equation \ref{eq:v_0L}. 
\begin{equation}
	\gls{symb:v_0L} =  \frac{1}{n} \sum_{i=1}^{n} \hat{u}_i
	\label{eq:v_0L}	
\end{equation}  
Where $n$ is the number of beams. 

\noindent This static \gls{WFR} is already providing a value that is comparable to a \gls{REWS}. The only drawback is, that it is related to a different point in space than the \gls{WT} rotor. Therefore the two values at the same point in time $t$ are not directly comparable: $\gls{symb:v_0L}(t) \neq \gls{symb:v_0}(t)$. This issue is tackled in the next section \ref{subsec:DynamicWFR}.    

\subsection{Dynamic Wind Field Reconstruction}\label{subsec:DynamicWFR}
As the word dynamic indicates the change to the static \gls{WFR} is, that the temporal relationship between the measurement and the effect on the \gls{WT} rotor itself is included. The basic concept behind this is Taylor’s Frozen Turbulence Hypothesis \cite{Taylor1938}. Which main take away is that the wind field is traveling with its mean wind speed \gls{symb:u_mean} through space. This leads to a time a wind field at a specific point upstream of the \gls{WT} needs until it reaches the rotor. Setting this point in space to the measurement point of the \gls{lidar} at distance \gls{symb:xL} leading to a travel time after Taylor of equation \ref{eq:T_taylor}
\begin{equation}
	\gls{symb:T_Taylor} =  \frac{\gls{symb:xL}}{\gls{symb:u_mean}} 
	\label{eq:T_taylor}	
\end{equation}
If the \gls{lidar} signal is shifted forward in time by \gls{symb:T_Taylor} the two values of $\gls{symb:v_0}(t)$ and $\gls{symb:v_0L}(t+\gls{symb:T_Taylor})$ are no directly comparable. 

\noindent To align the \gls{symb:v_0L} with the \gls{symb:v_0} as close as possible in phase as well es magnitude \cite{Schlipf2011} is already proposing the method of filtering and buffering. In addition shear and other effects can be included into the dynamic \gls{WFR}. For this thesis is only focusing on the dynamic step of the \gls{WFR}.

   
\section{Adaptive Data Driven Algorithms}\label{sec:AlgBasics}
Referring to \cite{Zaknich2005} an adaptive algorithm is an algorithm that can change its behavior during runtime. This change is driven by data and evaluated via a cost function. All developed algorithm need to be able to change there properties and this based on data to be called adaptive data driven algorithms. 