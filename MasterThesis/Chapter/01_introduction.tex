% !TEX spellcheck = en_US
\chapter{Introduction} \label{cha:INT}
\section{Motivation}\label{sec:Mot}
As the wind is the driving force of \gls{WT} it is important to measure it correctly. Nowadays the \gls{lidar} technology is used in wind industry as a wind sensor. Use cases for a \gls{lidar} are e.g. the site assessment and the \gls{LAC}. For \gls{LAC} the future wind is estimated by the \gls{lidar} measurement. This fundamental concept enables the \gls{FF} control concept of a \gls{WT} leading to load reduction and enhanced energy output !!!CITE!!!. The concept of \gls{LAC} includes next to design of a \gls{FF} controller the processing of the \gls{lidar} data itself. This includes understanding and modeling the behavior of the wind in a scale of sub hundred meters and several seconds. Altogether this aims to enhance our knowledge of the flow behavior in such scales as well es signal processing and system dynamics.

\section{Related Work}\label{sec:RelWork}
As this thesis works on the data processing of the \gls{lidar} it is not conducting any work on the controls itself nevertheless the most related work is still the \gls{LAC} based on the PhD Thesis by D.Schlipf \cite{Schlipf2015}.  